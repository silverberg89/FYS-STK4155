\section{Conclusion}
\setlength{\parindent}{0cm}
The use of machine learning for reproducing the election outcome gave valuable results that predicts the outcome with high certainty. All methods produced test results in the range of 85\% to 89\% for the 2012 election and in the range of 86\% to 95\% for the 2016 election. These results should be incorporated with the unbalanced target percentages of (79/21)\% and (87/13)\%, respectively. We can conclude that the SVM kernel method were slightly better suited than the Neural Network, and is an obvious choice if computational time were to be considered. The addition of the kernel trick raised the results substantially, considering the unbalance of the data. It is also apparent that the Neural Network is more sensitive for tuning parameters, as the results varies more than for SVM.  
\\
\par
Population size and age of the inhabitants are the two most important features for the classification of a democrat or republican. As for the flipped case we had the desire to sort out a fancy silver bullet that could explain the party shift in general. However, the disappointed results were to weak for any robust conclusion, only additional implications that the percentage of white people and the percentage of females may have had some extra impact.
\\
\par
We can also conclude that the accuracy score metric used throughout this project may not be the most suitable for this dataset, due to the described unbalance. The testing of the F1 score metric gave clear indications that further analysis should explore different metrics to tackle this dataset.