\section{Conclusion}
\setlength{\parindent}{0cm}
The use of machine learning for reproducing the election outcome gave valuable results that predicts the outcome with high certainty. All methods produced test results in the range of 85\% to 89\% for the 2012 election and in the range of 86\% to 95\% for the 2016 election. These results should be incorporated with the unbalanced target percentages of (79/21)\% and (87/13)\%, with respect to republicans and democrats. The first conclusion is that the SVM kernel method was slightly better suited than the Neural Network, and it's an obvious choice if computational time was to be considered. The addition of the kernel trick raised the results significantly when considering the unbalances in the data. We interpret the neural network as more sensitive for tuning parameters, since the results varies more than for SVM.  
\\
\par
Population size and age of the inhabitants are the two most important features for the classification of a democrat or republican county. In the flipped case, a desire to sort out a fancy silver bullet that could explain the party shift in general. However, the results were to weak for any decisive conclusion, only vague implications that the percentage of white people and the percentage of females may have had some extra impact.
\\
\par
Another big takeaway, is that the accuracy score metric used throughout this project, may not be the most suitable for this dataset. The testing of the F1 score metric in the variables analysis gave a clear indication that further analysis should explore different metrics to tackle this dataset. A manual reduction of the variables may also be of interest.