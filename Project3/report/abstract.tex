\section*{Abstract}
\setlength{\parindent}{0cm}
The report explores the 2012 and 2016 US election outcome based demographic and economic variables measured around the same time frame. The ultimate goal is to find out if it is possible to use the data to train a model on the 2012 election and then make predictions on the 2016 election. It’s also desirable to find out what variables that could explain why some counties changed the most popular party between the elections. Similar work has been done by many data scientists and political analytics firms recently. Ryan Peach’s data analysis work on the data is worth mentioning, and this report can be viewed as a spin-off \footnote{\url{https://i.stack.imgur.com/1fXzJ.png}}. 
\\
\par 

The dependent variables include ethnicity, education, population, retail sales and so on. The response variable is the percentage political turnout for the democratic party for every county. In order to make it a classification problem, which is more suitable in an election. Counties where the democrats got more than 50\% is labeled as 1, else 0.  
The models applied on the data are logistic regression, support vector machines and artificial neural network. All models have the ability of learn from the data and make good predictions.
\\
\par 

The work is based on the foundation of machine learning, which is data, a model and a cost function. The python programming language is used for implementation of the methods. The Python packages Sci-kit learn and Tensorflow and Keras are used extensively because of their rich machine learning functionality.